\eabstract{
Metagenomics is a discipline that directly extracts DNA sequences from environmental samples for research and exploration of microbial community-related issues. 
Binning is the process of classifying DNA contigs according to species attributes, and is a crucial step in current metagenomic research.
The challenges in effectively classifying metagenomic DNA contigs mainly include: first, current metagenomic binning methods do not fully utilize assembly graph information. 
Second, in metagenomic datasets, the varying lengths of each DNA contig lead to poor binning results.  
Therefore, this article focuses on some key aspects and challenges of the current DNA contig classification problem.

(1) The extracted features were clustered using a deep density clustering algorithm.

Using the feature vectors compressed by the variational graph autoencoder model in the second step, 
this study utilized the improved density-based spatial clustering of applications with noise (DBSCAN) algorithm to cluster metagenomic overlapping groups. 
In this algorithm, improvements based on locality-sensitive hashing (LSH) automatically determine the selection of the neighboring radius and sampling points, avoiding the manual input of potential clustering errors. 
Ultimately, the determined two parameters and feature vectors serve as inputs to complete the entire deep density clustering process.

(2) The proposed method was tested on datasets of varying complexities, and compared with other binning techniques based on three metrics: adjusted Rand index (ARI), precision, and recall.

This study tested the proposed method on datasets with different complexities, and compared it with other binning techniques using three metrics: adjusted Rand index (ARI), precision, and recall.
}{Metagenome binning, graph neural networks, variational autoencoders}

