\cabstract{
宏基因组学是直接从环境样本中提取 DNA 序列来进行研究,并以此对微生物的群落的相关问题进行研究和探索的学科。
分箱是指将 DNA 重叠群按物种的属性进行分类的过程,是当前宏基因组学研究中的关键步骤。
宏基因组 DNA 重叠群进行有效分类问题存在的难点主要有:一、当前的宏基因组分箱方法并没有充分利用装配图信息。二、在宏基因组数据集中,每条 DNA 重叠群长度长短不一导致分箱效果不好。
因此,本文对于目前 DNA 重叠群分类问题所存在的一些重点和难点方面进行了如下研究:

(1) 使用基于深度密度聚类算法对提取的特征进行聚类。

使用第二步通过变分图自编码器模型压缩后的特征向量,本文采用改进后的 DBSCAN(Density-Based Spatial Clustering of Applications with Noise)密度聚类算法,对宏基因组重叠群进行聚类分析。
在该算法中,通过基于局部敏感哈希(LSH)的改进,自动确定临近半径和采样点的选择,避免手动输入可能造成的聚类误差。
最终,将确定的两个参数和特征向量作为输入,完成整个深度密度聚类过程。

(2) 在不同复杂性的数据集上测试了本文提出的方法,就 ARI、精确率和召回率三个指标与其它分箱技术进行了比较。

}{宏基因组分箱,图神经网络,变分自编码器}

% Binning:从复杂的微生物体系中获取微生物基因组,测序技术不成熟的产物